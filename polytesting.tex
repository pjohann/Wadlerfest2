\documentclass{llncs}

\usepackage{stmaryrd}

\newcommand{\mbind}{\mathrel{>\kern-0.45em>\kern-0.45em=}}
\newcommand{\fold}[1]{\llparenthesis #1 \rrparenthesis} % FIXME: need to define this
\newcommand{\eqAnnotation}[1]{\hspace{2cm}\left\{\textrm{#1}\right\}}
\newcommand{\listappend}{\mathrel{+\kern-3pt+}}

\title{Testing Higher-Kinded Polymorphic Functions}
\author{Robert Atkey\inst{1}, Patricia Johann\inst{2}}

\institute{Computer and Information Sciences,
University of Strathclyde
\and
Computer Science Department,
Appalachian State University
}

\begin{document}

\maketitle

\begin{abstract}
  TBD
\end{abstract}

\section{Introduction}

Prof.~Phil Wadler has contributed profoundly to the science of
computer programming by taking deep ideas from the theoretical field
of denotational semantics and exposing them as programming
\emph{tools}. In \emph{Theorems for Free!}
\cite{DBLP:conf/fpca/Wadler89}, he used Reynolds' relational
parametricity to show that interesting theorems about polymorphic
functions can be deduced from their types. In \emph{The essence of
  functional programming} \cite{DBLP:conf/popl/Wadler92},
\emph{Comprehending monads} \cite{DBLP:journals/mscs/Wadler92}, and
\emph{Imperative functional programming}
\cite{DBLP:conf/popl/JonesW93} he showed how Moggi's use of monads to
capture computational effects provides a practical tool for
incorporating effects into a purely functional language.

In this paper, we combine these lines of work with the work of
Bernardy, Jansson and Claessen \cite{DBLP:conf/esop/BernardyJC10} on
parametricity-based testing of polymorphic functions. Since monads are
higher-kinded entities, this naturally leads us to the problem of
testing functions that are polymorphic at higher-kinds.

\subsection{Free Theorems}

In his famous paper \emph{Theorems for Free!}
\cite{DBLP:conf/fpca/Wadler89}, Phil Wadler made the following claim:
\begin{quotation}
  \noindent
  Write down the definition of a polymorphic function on a piece of
  paper. Tell me its type, but be careful not to let me see the
  function's definition. I will tell you a theorem that the function
  satisfies.
\end{quotation}
Phil then goes on to make good on his claim, exploiting Reynolds'
theory of relational parametricity \cite{DBLP:conf/ifip/Reynolds83} to
derive interesting properties of a range of polymorphic functions. His
initial example is for a function $r$ with polymorphic type:
\begin{displaymath}
  \mathit{r} :: \forall \alpha.~[\alpha] \to [\alpha]
\end{displaymath}
Just as Phil claims, it is possible, solely from the type of this
function, to deduce the following theorem: for all types $\alpha$ and
$\beta$, functions $f :: \alpha \to \beta$, and lists
$\mathit{xs} :: [\alpha]$,
\begin{displaymath}
  \mathit{map}~f~(\mathit{r}~\mathit{xs}) \equiv \mathit{r}~(\mathit{map}~f~\mathit{xs}).
\end{displaymath}
Since $\mathit{r}$ must operate the same way at whatever type $\alpha$
it is instantiated with, it does not matter whether we transform
$\alpha$ values to $\beta$ values via $f$ before or after applying
$\mathit{r}$. Thus we have a kind of invariance property of any
function $r$ of type $\forall \alpha.~[\alpha] \to [\alpha]$.

However, for programming, we are usually interested in specific
functions of certain types. For example, the function $r$ may be named
$\mathit{reverse}$, and we would expect it to actually reverse
lists. The fact that $\mathit{reverse}$ commutes with $\mathit{map}~f$
is useful to know, but it does not assure us that $\mathit{reverse}$
actually does what it name suggests. Infinitely many functions with
the same type as $\mathit{reverse}$ satisfy the property of commuting
with $\mathit{map}~f$. For example, the identity function and the
function that only reverses lists of even length both do.

It would seem now that we need to actually look at the definition of
$\mathit{reverse}$ to check that it actually does reverse lists,
reasoning by induction on the length of the input list, and so forth.

\subsection{Testing Polymorphic Functions}

However, there is another, more lightweight, approach: we could
\emph{test} the function $\mathit{reverse}$ on different inputs to
check that it actually reverses lists. Following the example of
QuickCheck \cite{DBLP:conf/icfp/ClaessenH00}, we could test
$\mathit{reverse}$ on a number of inputs, choosing those inputs at
random to make sure our tests are as unbiased as possible. In general,
testing cannot achieve the same degree of guarantee that formal proof
can, but it is often much cheaper to apply. Moreover, it can help to
discover defects before embarking on a formal proof (FIXME: Isabelle's
Nitpick).

Using a QuickCheck style approach, we test our $\mathit{reverse}$
function against a reference implementation on randomly generated
lists. We want to test the following property, for arbitrary
$\mathit{xs} :: [\alpha]$,
\begin{equation}\label{eq:reverse-test}
  \mathit{reverse}~\mathit{xs} \equiv \mathit{revSpec}~\mathit{xs}
\end{equation}
where $\mathit{revSpec}$ is defined as the inefficient (it has
quadratic complexity), but ``obviously'' correct recursive
implementation of list reversal:
\begin{displaymath}
  \begin{array}{l@{\hspace{0.3em}}l@{\hspace{0.3em}}cl}
    \mathit{revSpec}& [] & = & [] \\
    \mathit{revSpec}& (x\mathord:xs) & = & \mathit{revSpec}~\mathit{xs} \listappend [x]
  \end{array}
\end{displaymath}
However, we immediately run into a problem: the function
$\mathit{reverse}$ is polymorphic! It effectively takes two inputs: a
type $\alpha$ and a list of values of type $\alpha$. We cannot
generate random lists of an unknown type $\alpha$. What values would
we pick? Generating random types to instantiate $\alpha$ with also
will not work. If we randomly generate some complex function type,
then how should we generate random inhabitants of it?

Happily, there is a solution to this problem: we can use Phil's free
theorem to pick the a type instantiation that will stand in for all
possible types! The observation that this is possible is due to
Bernardy, Jansson and Claessen, in their paper \emph{Testing
  Polymorphic Properties} \cite{DBLP:conf/esop/BernardyJC10}. Their
reasoning goes as follows.

Assume for the moment that we are only interested in testing on lists
$\mathit{xs}$ of length $n$, but still of some arbitrary type
$\alpha$. We can form the type $\mathit{Fin}~n$ of natural numbers
less than $n$, which represent positions in lists of length
$n$. Letting
$\mathit{positions}_n = [0, 1, ..., n-1] :: [\mathit{Fin}~n]$ and
letting $f :: \mathit{Fin}~n \to \alpha$ be the list indexing
function, we now have that
$\mathit{xs} = \mathit{map}~f~\mathit{positions}_n$. Applying free
theorem for the type $\forall \alpha.~[\alpha] \to [\alpha]$ to both
sides of the property (\ref{eq:reverse-test}) that we want to test, we
have:
\begin{displaymath}
  \begin{array}{l}
    \mathit{reverse}~\mathit{xs} \equiv \mathit{reverse}~(\mathit{map}~f~\mathit{positions}_n) \equiv \mathit{map}~f~(\mathit{reverse}~\mathit{positions}_n) \\
    \mathit{revSpec}~\mathit{xs} \equiv \mathit{revSpec}~(\mathit{map}~f~\mathit{positions}_n) \equiv \mathit{map}~f~(\mathit{revSpec}~\mathit{positions}_n) \\
  \end{array}
\end{displaymath}
Thus, to check
$\mathit{reverse}~\mathit{xs} \equiv \mathit{revSpec}~\mathit{xs}$,
for lists $\mathit{xs}$ of length $n$ and arbitrary element type
$\alpha$, it suffices to show that
$\mathit{reverse}~\mathit{positions}_n \equiv
\mathit{revSpec}~\mathit{positions}_n$.

To test the property for all $n$, we use a QuickCheck-style approach
to randomly generate lengths $n$, execute both sides of the equation
and check for equality. By the free theorem, and the reasoning above,
if there is a counterexample to the correctness of reverse, then we
can find it by finding the list $\mathit{positions}_n$ on which
$\mathit{reverse}$ fails to act like $\mathit{revSpec}$.

\subsection{Testing Monadic Functions}

Another of Phil's contributions to the science of computer programming
was the introduction of \emph{monads} as a practical programming tool
for incorporating effects into a functional programming language,
taking them from Moggi's original use as a structuring device for
denotational semantics for those languages
\cite{DBLP:conf/lics/Moggi89}.

A key useful property of monads for programming is that there is more
than one monad. Phil showed how we can use the $\mathit{State}$ monad
if we want to write stateful programs, the $\mathit{List}$ monad if we
want to write non-deterministic programs, the $\mathit{Writer}$ monad
if we want to write programs that do logging, and so on.

The fact that there exists more than one monad means that it natural
to want to write programs that polymorphic over all monads. One
example is the $\mathit{sequence}$ function from the Haskell standard
library:
\begin{displaymath}
  \mathit{sequence} :: \forall m~a.~\mathit{Monad}~m \Rightarrow [m~a] \to m~[a]
\end{displaymath}
The intended behaviour of the $\mathit{sequence}$ function is to take
a list of monadic actions and to sequence them to produce a single
monadic action that yields a list of values. The function
$\mathit{sequence}$ is polymorphic in the monad $m$.

The polymorphism involved in functions like $\mathit{sequence}$ is
\emph{higher-kinded}. Unlike $\mathit{reverse}$, which was polymorphic
over types, $\mathit{sequence}$ is polymorphic over type operators. A
type operator is a type-level function from types to types. To keep
things straight, a kinding system is used in Haskell to separate types
from type operators. Types have kind $*$, while type operators have
kind $* \to *$. Higher kinds, like $(* \to *) \to *$ and
$(* \to *) \to (* \to *)$ are also possible.

Continuing the discussion of free theorems and testing from above, the
two immediate questions are: what are the free theorems for functions
like $\mathit{sequence}$? And how can we effectively test properties
of functions like $\mathit{sequence}$?

The first question has been addressed before. Voightl{\"a}nder
\cite{DBLP:journals/jfp/VytiniotisW10} examined an extension of free
theorems for a limited system with higher kinds, applying it to monads
in particular. Vytiniotis and Weirich
\cite{DBLP:journals/jfp/VytiniotisW10}, as well as the first author
\cite{atkey12relational}, gave parametricity theorems for all higher
kinded types in syntactic and denotational settings,
respectively. However, the higher kinded testing problem has not been
explored before, and that is the purpose of this paper.

\subsection{Overview}

% \bigskip

% Other related work:
% \begin{itemize}
% \item Harrison's ``Without Loss of Generality''
% \end{itemize}

\section{The Polymorphic Testing Problem}

The {\em polymorphic testing problem} considered by Bernardy, Jansson,
and Claessen can be formally stated as follows:

\begin{verse}\label{problem:poly-testing}
  \hspace*{0.2in}Let $\sigma[a]$ be a type expression with a free
  type variable $a$, and let $H$ be a definable functor. Given a
  pair of functions $h_1$, $h_2$ of type $\forall a :
  *.~\sigma[a] \to H a$, find a \emph{monomorphic} sufficient
  condition that implies the following property:
  \begin{equation}
    \label{eq:problem}
    \forall a : *. \forall x : \sigma[a].~h_1~[a]~x \equiv h_2~[a]~x
  \end{equation}
\end{verse}

\begin{example}
  We illustrate the polymorphic testing problem with an example:

\end{example}

The polymorphic testing problem was introduced by Bernardy, Jansson,
and Claessen~\cite{bjc10}, whose main result is:

\begin{theorem}\label{thm:poly-testing}
Let $\sigma[\alpha]$, $H$, $h_1$ and $h_2$ be as in the description of
the polymorphic testing problem. If there exist functors $\{G_i\}_{i
  \in I}$ and $F$, and types $\{O_i\}_{i \in I}$ such that
\begin{displaymath}
  \sigma[\alpha] = \left( \Pi_{i \in I} (G_i \alpha \to O_i)\right) 
  \times  (F \alpha \to \alpha)
\end{displaymath}
and if there exists an initial $F$-algebra $(\mu F, \mathrm{in} : F(\mu
F) \to \mu F)$, then the following condition is a solution for this
instance of the polymorphic testing problem:
\begin{equation}
  \label{eq:solution-pt}
  \forall p : \Pi_{i \in I}(G_i(\mu F) \to O_i).~h_1~(\mu F)~(p,
  \mathrm{in}) = h_2~(\mu F)~(p, \mathrm{in})
\end{equation}
\end{theorem}

We will say that a type of the form of $\sigma$ is in {\em BJC
  canonical form}.  The proof that (\ref{eq:solution-pt}) is a
solution to the polymorphic testing problem relies on the following
lemma. It is a consequence of the parametricity property for
polymorphic functions whose domains are in BJC canonical form.
\begin{lemma}\label{lem:initial-algebra-ok}
  Let $h : \forall \alpha : *. \left(\Pi_{i \in I} (G_i \alpha \to
  O_i)\right) \times (F \alpha \to \alpha) \to H\alpha$. Assume that
  there is an initial $F$-algebra $(\mu F, \mathrm{in} : F(\mu F) \to
  \mu F)$. If $\fold{f}$ is the unique $F$-algebra homomorphism from
  $(\mu F, \mathrm{in})$ to $(\alpha, f)$, then
  \begin{displaymath}
    \begin{array}{l}
      \forall \alpha : *. \forall p : \Pi_{i \in I} (G_i \alpha \to O_i). \forall f : F\alpha \to \alpha.\\
      \quad\quad\quad h~\alpha~(p, f) = H~\fold{f}~(h~\mu F~(\langle p_i \circ G_i\fold{f} \rangle_{i \in I}, \mathrm{in}))
    \end{array}
  \end{displaymath}
\end{lemma}

\begin{proof}
  We invoke the parametricity property for $h$ (specialised to the case
  of functional relations):
  \begin{equation}
    \label{eq:h-parametricity-prop}
    \begin{array}{l}
      \forall \alpha_1, \alpha_2 : *.~\forall r : \alpha_1 \to \alpha_2. \\
      \quad \forall p' : \Pi_{i \in I}(G_i\alpha_1 \to O_i), p : \Pi_{i \in I}(G_i\alpha_2 \to O_i). (\forall i \in I.~p'_i = p_i \circ G_i r) \Rightarrow \\
      \quad\quad\quad \forall f_1 : F\alpha_1 \to \alpha_1, f_2 : F\alpha_2 \to \alpha_2. r \circ f_1 = f_2 \circ F r \Rightarrow \\
      \quad\quad\quad\quad\quad H~r~(h~\alpha_1~(p',f_1)) =
      h~\alpha_2~(p,f_2) 
    \end{array}
  \end{equation}
  Given $\alpha : *$, $p : \Pi_{i \in I} (G_i\alpha \to O_i)$ and $f :
  F\alpha \to \alpha$, we instantiate (\ref{eq:h-parametricity-prop})
  with $\alpha_1 = \mu F$, $\alpha_2 = \alpha$, $r = \fold{f}$, $p' =
  \langle p_i \circ G_i \fold{f} \rangle_{i \in I}$, $p = p$, $f_1 =
  \mathrm{in}$ and $f_2 = f$. The condition on $p'$ and $p$ in the
  third line holds by definition, and the condition on $f_1$ and $f_2$
  in the fourth and fifth lines holds by the fact that $\fold{f}$ is
  an $F$-algebra homomorphism.
\end{proof}

We can use Lemma~\ref{lem:initial-algebra-ok} to prove that
(\ref{eq:solution-pt}) is a solution to the polymorphic testing
problem.

\begin{proof}
(Theorem~\ref{thm:poly-testing}) We want to prove that
  (\ref{eq:solution-pt}) implies the following specialization of
  (\ref{eq:problem}) to the current situation: 
  \begin{displaymath}
    \forall \alpha : *. \forall p : \Pi_{i \in I}(G_i\alpha \to
    O_i). \forall f : F\alpha \to \alpha.~h_1~\alpha~(p,f) =
    h_2~\alpha~(p,f)
  \end{displaymath}
  Given $\alpha : *$, $p : \Pi_{i \in I}(G_i\alpha \to O_i)$ and $f :
  F\alpha \to \alpha$, we reason as follows:
  \begin{displaymath}
    \begin{array}{cl}
       & h_1~\alpha~(p,f) \\
       =&\eqAnnotation{Lemma \ref{lem:initial-algebra-ok} applied to $h_1$} \\
       &H~\fold{f}~(h_1~\mu F~(\langle p_i \circ G_i\fold{f}
       \rangle_{i \in I}, \mathrm{in})) \\ 
       =&\eqAnnotation{Application of (\ref{eq:solution-pt})} \\
       &H~\fold{f}~(h_2~\mu F~(\langle p_i \circ G_i\fold{f}
       \rangle_{i \in I}, \mathrm{in})) \\ 
       =&\eqAnnotation{Lemma \ref{lem:initial-algebra-ok} applied to
         $h_2$} \\ 
       &h_2~\alpha~(p,f)
    \end{array}
  \end{displaymath}
\end{proof}



% TODO (basically transplant Section 2 from \texttt{paper.tex})
% \begin{enumerate}
% \item Text: We recapitulate the work by Bernardy et al. to put our
%   work in context...
% \item State the polymorphic testing problem
% \item Give the basic BJC solution, reduction by embedding-projection
%   pairs, and the use of containers. (could maybe skip the embedding
%   projection pairs part, since we can now use datakinds in Haskell to
%   do containers for many types...)
% \item Polymorphic Testing with Constraints
% \end{enumerate}

\section{The \emph{Higher-Kinded} Polymorphic Testing Problem}

% INTRO: why do we want to test higher-kinded polymorphic functions?
% Well, because they are there, and because monads and applicative
% functors are very useful.

% Writing $F \rightarrow G$ for $\forall \alpha : *.\ F \alpha
% \rightarrow G \alpha$ for (first-order) functors $F$ and $G$, we can
% state the {\em higher-order polymorphic testing problem} as:

% \begin{verse}\label{problem:hpoly-testing}
%   \hspace*{0.2in}Let $\sigma[L] : (* \rightarrow *) \rightarrow (*
%   \rightarrow *)$ be a map between type constructors with a free type
%   constructor variable $L$, and let $H : (* \rightarrow *) \rightarrow
%   (* \rightarrow *)$ be a definable higher-order functor. Given a pair
%   of functions $h_1$, $h_2 : \forall L: * \rightarrow *.\ \sigma[L]
%   \rightarrow HL$, find a \emph{monomorphic} sufficient condition that
%   implies the following property:
%   \begin{equation}
%     \label{eq:hproblem}
%     \forall L : * \rightarrow *. \forall x : \sigma[L].~h_1~L~x = h_2~L~x
%   \end{equation}
% \end{verse}

% FIXME: also formulate the higher-order poly testing problem with
% constraints...

% \subsection{Arbitrary Type Constructors}

% Give the useless instantiation...

% \subsection{Quantification over Monads}

FIXME: Introductory paragraph, explaining the purpose and structure of
this section.

% \begin{enumerate}
% \item Solution of the higher-kinded polymorphic testing problem for
%   monads, where the type is of the form:
%   \begin{displaymath}
%     \forall m~a. \mathrm{Monad}~m\Rightarrow \Sigma s : S.~(\forall b.~F_s~m~b \to m~b) \to H~m~a
%   \end{displaymath}
%   FIXME: this doesn't work quite right, because we can't always
%   extract the free monad operations out of this type.
% \item Similarly, solution of the higher-kinded polymorphic testing
%   problem for applicative functors, where the type is of the form:

% \end{enumerate}

Let us say we want to test a property of the standard Haskell
$\mathit{sequence}$ function, which has the following type:
% FIXME: check this type: the Haskell standard library has been
% generified recently
\begin{displaymath}
  \mathit{sequence} :: \forall m~a.~\mathit{Monad}~m \Rightarrow [m~a] \to m~[a]
\end{displaymath}
The intended behaviour of the $\mathit{sequence}$ function is to take
a list of monadic actions, in some arbitrary monad $m$, and to
sequence them to produce a single monadic action that yields a list of
values. An implied property of this informal specification is that
$\mathit{sequence}$ commutes with list append:
\begin{equation}\label{eq:sequence-append}
  \begin{array}{l}
    \forall m~a~\mathit{xs}~\mathit{ys}.~\mathit{Monad}~m \Rightarrow \\
    \quad \mathbf{do}~\{~\mathit{xs'} \leftarrow \mathit{sequence}~\mathit{xs}; \mathit{ys'} \leftarrow \mathit{sequence}~\mathit{ys}; \mathit{return}~(\mathit{xs'}\listappend\mathit{ys'}) \} \\
    \quad \equiv \mathit{sequence} (\mathit{xs}\listappend\mathit{ys})
  \end{array}
\end{equation}
Testing this property does not fall into the basic polymorphic testing
problem considered by Bernardy et al., due the quantification over all
\emph{monads}, which implies quantification over type constructors: in
this case, quantification ove the type variable $m$ of kind $* \to *$.

Abstracting out the two sides of equation (\ref{eq:sequence-append})
as functions, we have a pair of functions of type:
\begin{displaymath}
  h_1, h_2 :: \forall m~a. \mathit{Monad}~m \Rightarrow [m~a] \to [m~a] \to m~[a]
\end{displaymath}
where
\begin{displaymath}
  \begin{array}{lcl}
    h_1&=&\lambda \mathit{xs}~\mathit{ys} \to \mathbf{do}~\{~\mathit{xs'} \leftarrow \mathit{sequence}~\mathit{xs}; \mathit{ys'} \leftarrow \mathit{sequence}~\mathit{ys}; \mathit{return}~(\mathit{xs'}\listappend\mathit{ys'}) \} \\
    h_2&=&\lambda \mathit{xs}~\mathit{ys} \to \mathit{sequence}~(\mathit{xs}\listappend\mathit{ys})
  \end{array}
\end{displaymath}
Due to the quantification over all monads, testing equation
(\ref{eq:sequence-append}), i.e., the equivalence of $h_1$ and $h_2$
for a given implementation of $\mathit{sequence}$, is an instance of
the \emph{higher-kinded polymorphic testing problem} described
above. %FIXME: not really, due to the additional constraints
The problem is to find an instantiation for the type constructor
parameter $m$ that characterises the behaviour of the program on all
possible instantiations.

FIXME: we step through a similar sequences of steps to try and solve
the problem in this instance as worked for the base-kinded polymorphic
testing problem. We transform the type in several steps so that we can
find an instantiation of 

Spelling out the types involved in the $\mathit{Monad}$ typeclass
constraint, the type of $h_1$ and $h_2$ can be written as:
\begin{displaymath}
  \forall m~a.
  (\forall a. a \to m~a) \to
  (\forall a~b. m~a \to (a \to m~b) \to m~b) \to
  [m~a] \to
  [m~a] \to
  m~[a]
\end{displaymath}
Converting the lists in the argument positions to containers, and
performing currying to turn the $\Sigma$-bound natural numbers
determining the lengths into $\forall$-bound naturals, yields:
\begin{displaymath}
  \begin{array}{l}
    \forall m~a~n_1~n_2. \\
    \quad (\forall a. a \to m~a) \to (\forall a~b. m~a \to (a \to m~b) \to m~b) \to \\
    \quad (\mathit{Fin}~n_1 \to m~a) \to (\mathit{Fin}~n_2 \to m~a) \to \\
    \quad m~[a]
  \end{array}
\end{displaymath}
This type now looks familiar: there are four arguments to the
function, each of which has a type that has the form of a constructor
for values of type $m~a$, for some $a$. We might then believe that it
would be possible to follow the same approach as for the base-kinded
approach above and instantiate $m$ with the least fixpoint of the
functor describing these constructors. A suitable functor is the
following, described as a Haskell GADT\footnote{Is this a functor;
  need left-kan extensions?}:
\begin{displaymath}
  \begin{array}{l}
    \mathbf{data}~F~n_1~n_2~f~a~\mathbf{where} \\
    \quad
    \begin{array}{lcl}
      \mathsf{Return}&::&a \to F~n_1~n_2~f~a \\
      \mathsf{Bind}  &::&f~a \to (a \to f~b) \to F~n_1~n_2~f~a \\
      \mathsf{Oper1} &::&\mathit{Fin}~n_1 \to F~n_1~n_2~f~a \\
      \mathsf{Oper2} &::&\mathit{Fin}~n_2 \to F~n_1~n_2~f~a
    \end{array}
  \end{array}
\end{displaymath}
We can now take the least fixpoint of this functor using the
$\mathit{HFix}$ operator of kind
$((* \to *) \to (* \to *)) \to * \to *$, which is defined as follows:
\begin{displaymath}
  \mathbf{data}~\mathit{HFix}~f~a = \mathsf{HIn}~\{ \mathit{unHIn} :: f~(\mathit{HFix}~f)~a \}
\end{displaymath}
To use $\mathit{HFix}~(F~n_1~n_2)$ as the monomorphising instantiation
of our testing type, we need to be able to satisfy the
$\mathit{Monad}$ typeclass constraint, and to provide implementations
for the $\mathit{Fin}~n_1 \to m~a$ and $\mathit{Fin}~n_2 \to m~a$
arguments. A $\mathit{Monad}$ instance for the type constructor
$\mathit{HFix}~(F~n_1~n_2)$ is defined as follows:
\begin{displaymath}
  \begin{array}{l}
    \mathbf{instance}~\mathit{Monad}~(\mathit{HFix}~(F~n_1~n_2))~\mathbf{where} \\
    \quad
    \begin{array}{lcl}
      \mathit{return}  &=& \mathsf{HIn} \circ \mathsf{Return} \\
      c \mbind f &=& FIXME
    \end{array}
  \end{array}
\end{displaymath}
We can now start to see the first problem with our current path: this
instantiation of the operations of the $\mathit{Monad}$ typeclass do
not satisfy the monad laws! This is not necessarily a problem for some
testing problems (FIXME: refer back to discussion above on
constraints). However, for our problem, it is a problem, because it is
difficult to see how we will get an equality between $h_1$ and $h_2$
without at least assuming associaitivity of $(\mbind)$.

Another problem is that 



% Nevertheless, we can now reduce the testing problem to:

%   forall n1 n2.
%       h1 (map Oper1 [0..n1-1]) (map Oper2 [0..n2-1])
%    == h2 (map Oper1 [0..n1-1]) (map Oper2 [0..n2-1])

% which is monomorphic.

% Unfortunately, this has two problems:
%   1. The more serious one: equality on values of type 'HFix (F n) a' isn't decidable due to the function argument to Bind.
%   2. The property isn't true when the monad laws don't hold, because it relies on associativity of (>>=).

We can fix both of these by using free monad instead of
$\mathit{HFix}~(F~n_1~n_2)$. We define:
\begin{displaymath}
  \begin{array}{l}
    \mathsf{data}~M~n_1~n_2~a \\
    \quad
    \begin{array}{cl}
      = & \mathsf{Return}~a \\
      | & \mathsf{Oper1}~(\mathit{Fin}~n_1)~(M~n_1~n_2~a) \\
      | & \mathsf{Oper2}~(\mathit{Fin}~n_2)~(M~n_1~n_2~a)
    \end{array}
  \end{array}
\end{displaymath}
The type constructor $M~n_1~n_2$ can be endowed with a legitimate
$\mathit{Monad}$ instance:
\begin{displaymath}
  \begin{array}{l}
    \mathbf{instance}~\mathit{Monad}~(\mathit{HFix}~(F~n_1~n_2))~\mathbf{where} \\
    \quad
    \begin{array}{lcl}
      \mathit{return}             &=& \mathsf{Return} \\
      \mathsf{Return}~a \mbind f  &=& f~a \\
      \mathsf{Oper1}~i~c \mbind f &=& \mathsf{Oper1}~i~(c \mbind f) \\
      \mathsf{Oper2}~i~c \mbind f &=& \mathsf{Oper2}~i~(c \mbind f) \\
    \end{array}
  \end{array}
\end{displaymath}

%   oper1 i = Oper1 i (Return ())
%   oper2 i = Oper2 i (Return ())

% Equality on values of type (M n a) is decidable.

% Then we can reduce the testing problem to:

%   forall n1 n2.
%       h1 (map oper1 [0..n1-1]) (map oper2 [0..n2-1])
%    == h2 (map oper1 [0..n1-1]) (map oper2 [0..n2-1])

% Which is (a) monomorphic; and (b) decidable. We can exhaustively test this property for values of n1 and n2 up to about 20.

FIXME: formulate the higher-kinded polymorphic testing problem for
monads, and give the solution in terms of free monads.

\subsection{Testing Non-deterministic Functions}

MonadPlus

\subsection{Testing Imperative Functions}

Adaptation of the Claessen and Hughes Queue example...

\subsection{Functions that are Polymorphic over Applicative Functors}

Applicative functors \cite{mcbride-patterson} are a weakening of the
monad interface. By weakening the demands of the interface, there are
more applicative functors than there are monads, at the cost of a more
restrictive notion of `effectful program'.

Formulate the higher-kinded polymorphic testing problem for
applicative functors.

Just as we used the 

Capriotti and Kaposi gave a presentation of the free applicative
functor for a given functor...

\section{Conclusion}

In essence, the solution of many polymorphic testing problems is
straightforward: to test a program that quantifies over all $X$s, to
get a monomorphic instance for testing, we simply instantiate the
program with the ``free $X$''. We have presented several ways of 

Further work:
\begin{itemize}
\item Other uniformity principles --- takes a natural number
  parameter, but ``essentially'' does the same thing for all numbers;
  this would yield a way of testing some programs exhaustively, once
  we know their type.
\end{itemize}

\bibliography{polytesting} \bibliographystyle{plain}

\end{document}
