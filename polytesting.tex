\documentclass{llncs}

\newcommand{\mbind}{\mathrel{>\kern-0.45em>\kern-0.45em=}}

\title{Testing Higher-Kinded Polymorphic Functions}
\author{Robert Atkey\inst{1}, Patricia Johann\inst{2}}

\institute{Computer and Information Sciences,
University of Strathclyde
\and
Computer Science Department,
Appalachian State University
}

\begin{document}

\maketitle

\begin{abstract}
  TBD
\end{abstract}

\section{Introduction}

In his paper \emph{Theorems for Free}, Phil Wadler made the following
claim:
\begin{quotation}
  XXX
\end{quotation}
Characterise free theorems as essentially ``invariance'' properties of
programs with polymorphic types.

FIXME: Introduce the basic idea behind the BJC approach to polymorphic
testing, with an example.

Other related work:
\begin{itemize}
\item Harrison's ``Without Loss of Generality''
\end{itemize}

Our contribution: extension of the BJC approach to higher kinded
polymorphism: type constructor classes, which opens the possibility of
testing imperative programs.

\section{The Polymorphic Testing Problem}

TODO (basically transplant Section 2 from \texttt{paper.tex})
\begin{enumerate}
\item Text: We recapitulate the work by Bernardy et al. to put our
  work in context...
\item State the polymorphic testing problem
\item Give the basic BJC solution, reduction by embedding-projection
  pairs, and the use of containers. (could maybe skip the embedding
  projection pairs part, since we can now use datakinds in Haskell to
  do containers for many types...)
\item Polymorphic Testing with Constraints
\end{enumerate}

\section{The \emph{Higher-Kinded} Polymorphic Testing Problem}

INTRO: why do we want to test higher-kinded polymorphic functions?
Well, because they are there, and because monads and applicative
functors are very useful.

Writing $F \rightarrow G$ for $\forall \alpha : *.\ F \alpha
\rightarrow G \alpha$ for (first-order) functors $F$ and $G$, we can
state the {\em higher-order polymorphic testing problem} as:

\begin{verse}\label{problem:hpoly-testing}
  \hspace*{0.2in}Let $\sigma[L] : (* \rightarrow *) \rightarrow (*
  \rightarrow *)$ be a map between type constructors with a free type
  constructor variable $L$, and let $H : (* \rightarrow *) \rightarrow
  (* \rightarrow *)$ be a definable higher-order functor. Given a pair
  of functions $h_1$, $h_2 : \forall L: * \rightarrow *.\ \sigma[L]
  \rightarrow HL$, find a \emph{monomorphic} sufficient condition that
  implies the following property:
  \begin{equation}
    \label{eq:hproblem}
    \forall L : * \rightarrow *. \forall x : \sigma[L].~h_1~L~x = h_2~L~x
  \end{equation}
\end{verse}

FIXME: also formulate the higher-order poly testing problem with
constraints...

% \subsection{Arbitrary Type Constructors}

% Give the useless instantiation...

\subsection{Quantification over Monads}

% \begin{enumerate}
% \item Solution of the higher-kinded polymorphic testing problem for
%   monads, where the type is of the form:
%   \begin{displaymath}
%     \forall m~a. \mathrm{Monad}~m\Rightarrow \Sigma s : S.~(\forall b.~F_s~m~b \to m~b) \to H~m~a
%   \end{displaymath}
%   FIXME: this doesn't work quite right, because we can't always
%   extract the free monad operations out of this type.
% \item Similarly, solution of the higher-kinded polymorphic testing
%   problem for applicative functors, where the type is of the form:

% \end{enumerate}

Let us say we want to test the standard Haskell $\mathit{sequence}$
function, which has the following type:
% FIXME: check this type: the Haskell standard library has been
% generified recently
\begin{displaymath}
  \mathit{sequence} :: \forall m~a.~\mathit{Monad}~m \Rightarrow [m~a] \to m~[a]
\end{displaymath}
The $\mathit{sequence}$ function takes a list of monadic actions, in
some arbitrary monad $m$, and sequences them to produce a single
monadic action that yields a list of values. A useful property of
$\mathit{sequence}$ is that it commutes with list append:
\begin{equation}\label{eq:sequence-append}
  \begin{array}{l}
    \forall m~a~\mathit{xs}~\mathit{ys}.~\mathit{Monad}~m \Rightarrow \\
    \quad \mathbf{do}~\{~\mathit{xs'} \leftarrow \mathit{sequence}~\mathit{xs}; \mathit{ys'} \leftarrow \mathit{sequence}~\mathit{ys}; \mathit{return}~(\mathit{xs'}++\mathit{ys'}) \} \\
    \quad \equiv \mathit{sequence} (\mathit{xs}++\mathit{ys})
  \end{array}
\end{equation}
Abstracting out the two sides of equation (\ref{eq:sequence-append})
as functions, we have a pair of functions of type:
\begin{displaymath}
  h_1, h_2 :: \forall m~a. \mathit{Monad}~m \Rightarrow [m~a] \to [m~a] \to m~[a]
\end{displaymath}
where
\begin{displaymath}
  \begin{array}{lcl}
    h_1&=&\lambda \mathit{xs}~\mathit{ys} \to \mathbf{do}~\{~\mathit{xs'} \leftarrow \mathit{sequence}~\mathit{xs}; \mathit{ys'} \leftarrow \mathit{sequence}~\mathit{ys}; \mathit{return}~(\mathit{xs'}\mathit{+\kern-3pt+}\mathit{ys'}) \} \\
    h_2&=&\lambda \mathit{xs}~\mathit{ys} \to \mathit{sequence}~(\mathit{xs}\mathit{+\kern-3pt+}\mathit{ys})
  \end{array}
\end{displaymath}
Due to the quantification over all monads, testing equation
(\ref{eq:sequence-append}), i.e., the equivalence of $h_1$ and $h_2$
for a given implementation of $\mathit{sequence}$, is an instance of
the \emph{higher-kinded polymorphic testing problem} described
above. %FIXME: not really, due to the additional constraints
The problem is to find an instantiation for the type constructor
parameter $m$ that characterises the behaviour of the program on all
possible instantiations.

FIXME: we step through a similar sequences of steps to try and solve
the problem in this instance as worked for the base-kinded polymorphic
testing problem. We transform the type in several steps so that we can
find an instantiation of 

Spelling out the types involved in the $\mathit{Monad}$ typeclass
constraint, the type of $h_1$ and $h_2$ can be written as:
\begin{displaymath}
  \forall m~a.
  (\forall a. a \to m~a) \to
  (\forall a~b. m~a \to (a \to m~b) \to m~b) \to
  [m~a] \to
  [m~a] \to
  m~[a]
\end{displaymath}
Converting the lists in the argument positions to containers, and
performing currying to turn the $\Sigma$-bound natural numbers
determining the lengths into $\forall$-bound naturals, yields:
\begin{displaymath}
  \begin{array}{l}
    \forall m~a~n_1~n_2. \\
    \quad (\forall a. a \to m~a) \to (\forall a~b. m~a \to (a \to m~b) \to m~b) \to \\
    \quad (\mathit{Fin}~n_1 \to m~a) \to (\mathit{Fin}~n_2 \to m~a) \to \\
    \quad m~[a]
  \end{array}
\end{displaymath}
This type now looks familiar: there are four arguments to the
function, each of which has a type that has the form of a constructor
for values of type $m~a$, for some $a$. We might then believe that it
would be possible to follow the same approach as for the base-kinded
approach above and instantiate $m$ with the least fixpoint of the
functor describing these constructors. A suitable functor is the
following, described as a Haskell GADT\footnote{Is this a functor;
  need left-kan extensions?}:
\begin{displaymath}
  \begin{array}{l}
    \mathbf{data}~F~n_1~n_2~f~a~\mathbf{where} \\
    \quad
    \begin{array}{lcl}
      \mathsf{Return}&::&a \to F~n_1~n_2~f~a \\
      \mathsf{Bind}  &::&f~a \to (a \to f~b) \to F~n_1~n_2~f~a \\
      \mathsf{Oper1} &::&\mathit{Fin}~n_1 \to F~n_1~n_2~f~a \\
      \mathsf{Oper2} &::&\mathit{Fin}~n_2 \to F~n_1~n_2~f~a
    \end{array}
  \end{array}
\end{displaymath}
We can now take the least fixpoint of this functor using the
$\mathit{HFix}$ operator of kind
$((* \to *) \to (* \to *)) \to * \to *$, which is defined as follows:
\begin{displaymath}
  \mathbf{data}~\mathit{HFix}~f~a = \mathsf{HIn}~\{ \mathit{unHIn} :: f~(\mathit{HFix}~f)~a \}
\end{displaymath}
To use $\mathit{HFix}~(F~n_1~n_2)$ as the monomorphising instantiation
of our testing type, we need to be able to satisfy the
$\mathit{Monad}$ typeclass constraint, and to provide implementations
for the $\mathit{Fin}~n_1 \to m~a$ and $\mathit{Fin}~n_2 \to m~a$
arguments. A $\mathit{Monad}$ instance for the type constructor
$\mathit{HFix}~(F~n_1~n_2)$ is defined as follows:
\begin{displaymath}
  \begin{array}{l}
    \mathbf{instance}~\mathit{Monad}~(\mathit{HFix}~(F~n_1~n_2))~\mathbf{where} \\
    \quad
    \begin{array}{lcl}
      \mathit{return}  &=& \mathsf{HIn} \circ \mathsf{Return} \\
      c \mbind f &=& FIXME
    \end{array}
  \end{array}
\end{displaymath}
We can now start to see the first problem with our current path: this
instantiation of the operations of the $\mathit{Monad}$ typeclass do
not satisfy the monad laws! This is not necessarily a problem for some
testing problems (FIXME: refer back to discussion above on
constraints). However, for our problem, it is a problem, because it is
difficult to see how we will get an equality between $h_1$ and $h_2$
without at least assuming associaitivity of $(\mbind)$.

Another problem is that 



% Nevertheless, we can now reduce the testing problem to:

%   forall n1 n2.
%       h1 (map Oper1 [0..n1-1]) (map Oper2 [0..n2-1])
%    == h2 (map Oper1 [0..n1-1]) (map Oper2 [0..n2-1])

% which is monomorphic.

% Unfortunately, this has two problems:
%   1. The more serious one: equality on values of type 'HFix (F n) a' isn't decidable due to the function argument to Bind.
%   2. The property isn't true when the monad laws don't hold, because it relies on associativity of (>>=).

We can fix both of these by using free monad instead of
$\mathit{HFix}~(F~n_1~n_2)$. We define:
\begin{displaymath}
  \begin{array}{l}
    \mathsf{data}~M~n_1~n_2~a \\
    \quad
    \begin{array}{cl}
      = & \mathsf{Return}~a \\
      | & \mathsf{Oper1}~(\mathit{Fin}~n_1)~(M~n_1~n_2~a) \\
      | & \mathsf{Oper2}~(\mathit{Fin}~n_2)~(M~n_1~n_2~a)
    \end{array}
  \end{array}
\end{displaymath}
The type constructor $M~n_1~n_2$ can be endowed with a legitimate
$\mathit{Monad}$ instance:
\begin{displaymath}
  \begin{array}{l}
    \mathbf{instance}~\mathit{Monad}~(\mathit{HFix}~(F~n_1~n_2))~\mathbf{where} \\
    \quad
    \begin{array}{lcl}
      \mathit{return}             &=& \mathsf{Return} \\
      \mathsf{Return}~a \mbind f  &=& f~a \\
      \mathsf{Oper1}~i~c \mbind f &=& \mathsf{Oper1}~i~(c \mbind f) \\
      \mathsf{Oper2}~i~c \mbind f &=& \mathsf{Oper2}~i~(c \mbind f) \\
    \end{array}
  \end{array}
\end{displaymath}

%   oper1 i = Oper1 i (Return ())
%   oper2 i = Oper2 i (Return ())

% Equality on values of type (M n a) is decidable.

% Then we can reduce the testing problem to:

%   forall n1 n2.
%       h1 (map oper1 [0..n1-1]) (map oper2 [0..n2-1])
%    == h2 (map oper1 [0..n1-1]) (map oper2 [0..n2-1])

% Which is (a) monomorphic; and (b) decidable. We can exhaustively test this property for values of n1 and n2 up to about 20.

FIXME: formulate the higher-kinded polymorphic testing problem for
monads, and give the solution in terms of free monads.

\subsection{Testing Non-deterministic Functions}

\subsection{Testing Imperative Functions}

Adaptation of the Claessen and Hughes Queue example...

\subsection{Functions that are Polymorphic over Applicative Functors}

Applicative functors \cite{mcbride-patterson} are a weakening of the
monad interface. By weakening the demands of the interface, there are
more applicative functors than there are monads, at the cost of a more
restrictive notion of `effectful program'.

Formulate the higher-kinded polymorphic testing problem for
applicative functors.

Just as we used the 

Capriotti and Kaposi gave a presentation of the free applicative
functor for a given functor...

\section{Conclusion}

In essence, the solution of many polymorphic testing problems is
straightforward: to test a program that quantifies over all $X$s, to
get a monomorphic instance for testing, we simply instantiate the
program with the ``free $X$''. We have presented several ways of 

Further work:
\begin{itemize}
\item Other uniformity principles --- takes a natural number
  parameter, but ``essentially'' does the same thing for all numbers;
  this would yield a way of testing some programs exhaustively, once
  we know their type.
\end{itemize}

% \bibliography{polytesting} \bibliographystyle{plain}


\end{document}
